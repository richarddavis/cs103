%Modified from a template provided by Jennifer Pan, August 2011

\documentclass[10pt,letter]{article}
	% basic article document class
	% use percent signs to make comments to yourself -- they will not show up.
\usepackage{pdfsync}
\usepackage{amsmath}
\usepackage{amssymb}
\usepackage{amsthm}
	% packages that allow mathematical formatting

\usepackage{graphicx}
	% package that allows you to include graphics
\graphicspath{ {./images/} }


\usepackage{subcaption}

\usepackage{setspace}
	% package that allows you to change spacing

\onehalfspacing
	% text become 1.5 spaced

\usepackage{fullpage}
% package that specifies normal margins

\usepackage[parfill]{parskip}

\newtheorem*{thm}{Theorem}
\newtheorem{nthm}{Theorem}
\newtheorem{lem}{Lemma}

\begin{document}
	% line of code telling latex that your document is beginning

\title{Problem Set 3}

\author{Katherine Cheng, Richard Davis, Marty Keil}

% \date{Friday April 10, 2015}
	% Note: when you omit this command, the current date is automatically included
 
\maketitle 
	% tells latex to follow your header (e.g., title, author) commands.

\section*{Problem 1: Simplifying Propositional Formulas}
\paragraph{i)} $\neg p$. In this formula whenever p is false the entire statement is true. This is because p implies q in both sub-formulas, and anytime an antecedent is false, the entire statement is true. Conversely, whenever p is true, the entire statement is false. For instance, when p is true and q is true, the negation of q creates a false second statement. However, when p is true and q is false, the first statement is now false. Either way one is left with one false statement and since the "and" conjuction requires that both statements be true, the entire statement is always false in this scenario.

\paragraph{ii)} $\top$. In any case the result is true. For p not to imply q, p must true and q must be false. If we assume this then q must imply p because anytime q is false, the implication statement is true. Therefore the overall statement is true since it uses an "or" conjunction, which requires one true statement. 

\paragraph{iii)}  p $\leftrightarrow$ q.  Whenever p and q match the overall statement is true, otherwise the statement is false. When one variable is true, but the other is false, the first statement is true due to an "or" conjunction. Under these same conditions the second statement is false due to an "and" conjunction. With a true antecedent and false consequent, the entire formula is always false. However, when one considers identical truth values for p and q, the "or" and "and" conjunctions are no longer different, and the statement will always be true in this case. The bi-conditional conjunction satisfies this same restriction, requiring matching statement outcomes for a true result.  

\section*{Problem 2: Ternary Conditionals}
\paragraph{i)}
Truth Table \\
\newline
\begin{tabular}{ c | c | c | c }
  p & q & r &  {p ? q : r}\\
  \hline
  F & F & F & F \\
  F & T & F & F \\
  T & F & F & F \\
  T & F & T & F \\
  F & T & T & T \\
  F & F & T & T \\
  T & T & F & T \\
  T & T & T & T
\end{tabular}


\paragraph{ii)}
$ (p \wedge q) \vee (\neg p \wedge r)$ \\
  \newline
  Truth Table \\
  \newline
\begin{tabular}{ c | c | c | c }
  p & q & r &  {$ (p \wedge q) \vee (\neg p \wedge r)$}\\
  \hline
  F & F & F & F \\
  F & T & F & F \\
  T & F & F & F \\
  T & F & T & F \\
  F & T & T & T \\
  F & F & T & T \\
  T & T & F & T \\
  T & T & T & T
  
\end{tabular}
\paragraph{iii)}
 p ? $\bot : \top $

\paragraph{iv)}
 p ? q : $\top  $

\section*{Problem Three: First-Order Negations (definitely needs to be checked/edited)}

\begin{enumerate}
\item[i.] Negating the statement, ``For all x in the set of real numbers, and for all y in the set of real numbers, if x is less than y then there exists q in the set of rational numbers such that x is less than q, and q is less than y."
\begin{gather}
\neg(\forall x \in \mathbb{R}.\ \forall y \in \mathbb{R}. (x < y \rightarrow \exists q \in \mathbb{Q}. (x < q \wedge q < y)))\\
\exists x \in \mathbb{R}. \neg (\forall y \in \mathbb{R}. (x < y \rightarrow \exists q \in \mathbb{Q}. (x < q \wedge q < y)))\\
\exists x \in \mathbb{R}.\ \exists y \in \mathbb{R}. \neg(x < y \rightarrow \exists q \in \mathbb{Q}. (x < q \wedge q < y))\\
\exists x \in \mathbb{R}.\ \exists y \in \mathbb{R}. (x < y) \wedge \neg(\exists q \in \mathbb{Q}. (x < q \wedge q < y))\\
\exists x \in \mathbb{R}.\ \exists y \in \mathbb{R}. (x < y) \wedge \neg(\exists q \in \mathbb{Q}. (x < q \wedge q < y))\\
\exists x \in \mathbb{R}.\ \exists y \in \mathbb{R}. (x < y) \wedge \forall q \in \mathbb{Q}. \neg(x < q \wedge q < y))\\
\exists x \in \mathbb{R}.\ \exists y \in \mathbb{R}. (x < y) \wedge \forall q \in \mathbb{Q}. \neg(x < q) \vee \neg(q < y)\\
\exists x \in \mathbb{R}.\ \exists y \in \mathbb{R}. (x < y) \wedge \forall q \in \mathbb{Q}. (x \geq q) \vee (q \geq y)\\
\end{gather}
The negation is, ``There exists x in the set of real numbers, and there exists y in the set of real numbers, where x is less than y and for all q in the set of rational numbers x is greater than or equal to q, and q is greater than or equal to y."\\

\item[ii.] Negating the statement, ``For all x, y, and z, if (R(x, y) and R(y, z) then R(x, z)) then if for all x, y, and z R(y, z) and R(z, y) then R(z, x)."
\begin{gather}
\neg(\forall x. \forall y. \forall z. (R(x, y) \wedge R(y, z) \rightarrow R(x, z))) \rightarrow (\forall x. \forall y. \forall z. (R(y, x) \wedge R(z, y) \rightarrow R(z, x)))\\
\exists x. \exists y. \exists z. \neg((R(x, y) \wedge R(y, z) \rightarrow R(x, z))) \rightarrow (\forall x. \forall y. \forall z. (R(y, x) \wedge R(z, y) \rightarrow R(z, x)))\\
\exists x. \exists y. \exists z. ((R(x, y) \wedge R(y, z) \rightarrow R(x, z))) \wedge \neg(\forall x. \forall y. \forall z. (R(y, x) \wedge R(z, y) \rightarrow R(z, x)))\\
\exists x. \exists y. \exists z. ((R(x, y) \wedge R(y, z) \rightarrow R(x, z))) \wedge \exists x. \exists y. \exists z. \neg(R(y, x) \wedge R(z, y) \rightarrow R(z, x))\\
\exists x. \exists y. \exists z. ((R(x, y) \wedge R(y, z) \rightarrow R(x, z))) \wedge \exists x. \exists y. \exists z. \neg R(y, x) \vee \neg(R(z, y) \rightarrow R(z, x))\\
\exists x. \exists y. \exists z. ((R(x, y) \wedge R(y, z) \rightarrow R(x, z))) \wedge \exists x. \exists y. \exists z. \neg R(y, x) \vee R(z, y) \wedge \neg R(z, x))\\
\end{gather}
The negation is, ``There exists x, y, and z, where if R(x, y) and R(y, z) then R(x, z), and there exists x, y, and z such that not R(y, z) or R(z, y) and not R(z, x)."

\item[iii.] Negating the statement, ``For all x, there exists S such that S is a set and for all z, if z is in set S then z=x, and if z=x then z is in set S." 
\begin{gather}
\neg(\forall x. \exists S. (Set(S) \wedge \forall z. (z \in S \leftrightarrow z = x)))\\
\exists x. \neg(\exists S. (Set(S) \wedge \forall z. (z \in S \leftrightarrow z = x)))\\
\exists x. \forall S. \neg(Set(S) \wedge \forall z. (z \in S \leftrightarrow z = x))\\\exists x. \forall S. \neg Set(S) \vee \neg(\forall z. (z \in S \leftrightarrow z = x))\\\exists x. \forall S. \neg Set(S) \vee \exists z. \neg(z \in S \leftrightarrow z = x)\\\exists x. \forall S. \neg Set(S) \vee \exists z. \neg(z \in S \leftrightarrow z = x)\\
\end{gather}
How to distribute negation over bidirectional arrow?
\end{enumerate}

\section*{Problem 4: 'Cause I'm Happy}

\paragraph{i)} Statment 2. The only way to make a conditional statement false is if the antecedent is true and the consequent is false. In this case, when the antecedent is true everyone who is a person is happy. If this is true, then it is impossible that there exists a person who is unhappy. 

\paragraph{ii)} Statement 1. 

\paragraph{iii)} Statement 2. 

\paragraph{iv)} Statement 2. 

\paragraph{v)} Statement 2. 

\paragraph{vi)} Statement 2.

\section*{Problem 5: Translating into Logic}

\paragraph{i)} $\forall n Natural(n) .\ \exists k Natural(k) .\ \exists j Natural(j) .\ (n = (k + k) \cap (n * n) = (j + j))$

\paragraph{ii)} $ \exists$p. $\exists$k.  $\exists$j.(Person(p) $\wedge$ HasPet(p, kitten(k)) $\wedge$ HasPet(p,  kitten(j)) $\wedge$ $\forall$b.((b $\neq$ k  $\wedge$ b $\neq$   j) $\rightarrow$ ($\neg$ HasPet(p, B))))

\paragraph{iii)} The key to this problem is that an irrational number can not be represented as p/q, or $\sqrt{2} \not = p/q$. We can simplify this to $2 * q^2 \not = p^2$, then use the same method as in i).

\paragraph{iv)}
$\forall Q Set(Q) .\ \exists P Set(P) .\ \forall S Set(S) .\ (S \in P \cap \forall x .\ \forall y .\ (x \in S \rightarrow x \in Q \cap y \not \in S \rightarrow y \not \in Q))$

This statement translated into English says ``Any set Q has a powerset P that contains every subset S of Q.''
\paragraph{v)}
$\exists x .\ (Lady(x)\ \cap\ \forall y .\ (Glitters(y) \rightarrow IsSureIsGold(x, y))\ \cap\ \exists z .\ (StairwayToHeaven(z)\ \cap\ Buying(x, z)))$

\section*{Problem 6: Raven Paradox}
The Raven Paradox results from taking the seemingly uninteresting statement ``All ravens are black'' and finding its contrapositive ``Everything that is not black is not a raven.'' An important thing to note is that these statements are logically equivalent. Again, so far this is not especially interesting. The trouble arises when we start looking for phenomena that provide evidence for the statement. For instance, if you see a raven and that raven is black, this can be seen as providing evidence for the statement ``All ravens are black.'' However, because the contrapositive is logically equivalent to this statement, observing something that is not black and also not a raven must also be seen as consituting evidence. This is paradoxical because the two do not seem connected at all; looking at a red clown nose does not (intuitively) seem to give us any evidence about the blackness of ravens. 

In our discussions, we noted that the statement ``All ravens are black'' has the form of a scientific hypothesis. Everyone in the group had taken a course where we studied Popper and remembered what he had to say about scientific hypotheses: they can only be falsified, never confirmed. We have seen something similar in our dealings with first-order logic in this course. The statement ``All ravens are black'' can be turned into a statment in first-order logic with the form $\forall x .\ Raven(x) -> Black(x)$. This statement is only true if it is true for each and every raven. But no matter how many ravens we see that are black, it only takes a single albino raven to falsify the entire statement. 

This shows that one of the ways of resolving this paradox is to attack the idea of a phenomenon providing evidence for a statement. If induction is in fact a myth, then seeing a black raven provides the same amount of evidence as seeing a red clown nose: none. 

\section*{Problem 7: Graph Coloring}

\paragraph{i)}
In this graph the inner-node is connected to all other nodes, giving it a degree of seven. All other nodes are only connected to the inner node, which allow these nodes to all be of the same color and the inner node to be of a different color. This creates a two-colorable graph since no two nodes of the same color are joined by an edge. \\
\begin{minipage}{.8\textwidth}
\includegraphics[width=.8\linewidth]{hw3_slide1.jpg}
\end{minipage}

\paragraph{ii)}
This graph represents a cube, in which each corner shares an edge with three other corners. This means there is a degree of three for every node. It is possible for each corner of the cube to be connected to a corner with only one different color, createing a two-colorable graph. \\
\begin{minipage}{.8\textwidth}
\includegraphics[width=.8\linewidth]{hw3_slide2.jpg}
\end{minipage}

\paragraph{iii)}
In a triangle each node is connected to the other two nodes, meaning each node has a degree of 2. This graph is only 3-colorable because each node shares an edge with one another. If one tried to just use two colors, it is guaranteed that these two nodes would share an edge, violating the rule of a 2-colorable graph.  
\begin{minipage}{.8\textwidth}
\includegraphics[width=.8\linewidth]{hw3_slide3.jpg}
\end{minipage}

\pagebreak

\section*{Problem 8: Tournament Cycles}
A tournament is a directed graph with $n$ nodes where there is exactly one edge between any pair of distinct nodes and there are no self-loops. Prove that if a tournament graph contains a cycle of any length, then it contains a cycle of length three.

\begin{lem} \label{lem:q8_1} Given a directed graph with $n$ nodes, if the directed graph contains a cycle of length $k > 3$, then connecting any unconnected nodes in the cycle will create a cycle of length $< k$.
\end{lem}

\begin{proof} The definition of a cycle in a directed graph is a path from a node to itself. Pick an arbitrary cycle in a connected graph of length $k$ and pick any two arbtrary nodes in that cycle $m, n$ that are not already connected (i.e., not adjacent in the cycle). To connect $m$ and $n$ we can either place the base of the arrow at $m$ and the head at $n$ or the base at $n$ and the head at $m$. Assume we place the base at $m$ and the head at $n$. We know there already exists a cycle that starts at $m$, passes through at least one node that is not $n$, passes through $n$, and then passes through at least one node that is not $n$ before returning to $m$. By connecting $m$ and $n$ we create a new path that skips all the nodes between $m$ and $n$. Because we know there is at least one node between them, we create a new cycle that cuts out at least one node in the cycle and thus must be $< k$. We know that this is true no matter how we place the base and head by symmetry.
\end{proof}

\begin{thm} If a tournament graph contains a cycle of any length, then it contains a cycle of length three.
\end{thm}

\begin{proof} By strong induction. Let $P(n)$ be the statement ``if a tournament contains a cycle of length $n$, then it contains a cycle of length three.'' We will prove that $P(n)$ holds for $n \ge 3$. 

For our base case, we show that $P(3)$ is true. $P(3)$ states that a tournament with a cycle of length 3 contains a cycle of length 3. This is a tautology. 

For our inductive step, assume that for some $n \ge 3$, that for any $k \le n$, that $P(k)$ is true; that is, that a tournament containing a cycle of length $k$ also contains a cycle of length three. We will prove that $P(k+1)$ is true, that if a tournament contains a cycle of length $k+1$, then it also contains a cycle of length three.

Consider any tournament with a cycle of length $k+1$. Remove all the edges from this graph except for this cycle. A representation of this graph is shown in Figure \ref{fig:q8_cyc1}. Because this is a tournament, we know that every node in the cycle must be connected to every other node in the cycle. As soon as we draw an edge to reconnect two of the nodes we know by Lemma \ref{lem:q8_1} that we create a cycle of length $< k+1$ (see Figure \ref{fig:q8_cyc2} for illustration). By our inductive hypothesis we know that any cycle of length $\le k$ contains a cycle of length three. 

\begin{figure}[h]
\centering
\begin{minipage}{.5\textwidth}
  \centering
  \includegraphics[width=.8\linewidth]{hw3_8_1.eps}
  \captionof{figure}{A cycle of length $k+1$}
  \label{fig:q8_cyc1}
\end{minipage}%
\begin{minipage}{.5\textwidth}
  \centering
  \includegraphics[width=.8\linewidth]{hw3_8_2.eps}
  \captionof{figure}{Connecting two nodes in the cycle}
  \label{fig:q8_cyc2}
\end{minipage}
\end{figure}

\end{proof}

% \section*{Appendix: Referencing Equations}
% \begin{equation} \label{eq:divbyzero}
%   \frac {1} {0}
% \end{equation}

% This references \ref{eq:divbyzero}.

% \section*{Appendix: Figures in Text}
% Below are two different ways of placing figures side by side in text. The first method creates two sub-figures within a single figure. The second method creates two separate figures.

% \begin{figure}
% \centering
% \begin{minipage}{.5\textwidth}
%   \centering
%   \includegraphics[width=.8\linewidth]{hw3_8_1.eps}
%   \captionof{figure}{A figure}
%   \label{fig:q8_test1}
% \end{minipage}%
% \begin{minipage}{.5\textwidth}
%   \centering
%   \includegraphics[width=.8\linewidth]{hw3_8_1.eps}
%   \captionof{figure}{Another figure}
%   \label{fig:q8_test2}
% \end{minipage}
% \end{figure}


% \begin{figure}[h]
%   \centering

%   \begin{subfigure}[b]{0.3\textwidth}
%     \includegraphics[width=\textwidth]{hw3_8_1.eps}
%     \caption{A cycle with length $k+1$}
%     \label{fig:q8_cycle:a}
%   \end{subfigure}% 
%   \qquad
%   \begin{subfigure}[b]{0.3\textwidth}
%     \includegraphics[width=\textwidth]{hw3_8_1.eps}
%     \caption{A cycle with length $k+1$}
%     \label{fig:q8_cycle:b}
%   \end{subfigure}%  

%   \caption{Placeholder}
%   \label{fig:q8}
% \end{figure}

\end{document}
	% line of code telling latex that your document is ending. If you leave this out, you'll get an error

%%% Local Variables:
%%% mode: latex
%%% TeX-master: t
%%% End:
