%Modified from a template provided by Jennifer Pan, August 2011

\documentclass[10pt,letter]{article}
	% basic article document class
	% use percent signs to make comments to yourself -- they will not show up.
\usepackage{pdfsync}
\usepackage{amsmath}
\usepackage{amssymb}
\usepackage{amsthm}
\usepackage{ mathrsfs }
	% packages that allow mathematical formatting


\usepackage{graphicx}
	% package that allows you to include graphics

\usepackage{setspace}
	% package that allows you to change spacing

\onehalfspacing
	% text become 1.5 spaced

\usepackage{fullpage}
% package that specifies normal margins

\usepackage[parfill]{parskip}

\newtheorem*{thm}{Theorem}
\newtheorem{nthm}{Theorem}

\begin{document}
	% line of code telling latex that your document is beginning

\title{Problem Set 6: CS103}

\author{Marty Keil}

% \date{Friday April 10, 2015}
	% Note: when you omit this command, the current date is automatically included
 
\maketitle
	% tells latex to follow your header (e.g., title, author) commands.

\section*{Problem 1: The Myhill-Nerode Theorem}
\paragraph{i}
Proof: One can create a DFA for this example. Show DFA. If this is the DFA, then must be a regular language. 

\paragraph{ii}
Every odd number distinct string is distinct from any even length string over the language. Since there are infinite number of odd length strings and even number strings, and therefore even/odd pairs, then there must be infinite distinct pairs over the set. 

\paragraph{iii}
If it is an infinite set then you are going to have any even element strings not be distinct, or any odd element pairs not be distinct. 

\section*{Problem 2: Balanced Parentheses}
\paragraph{i}
\thm The Language L = ( $w \in ?*$ | w is a string of balanced parentheses ) is not regular. 

\proof Let $S = (,)*.$ This set is infinite because it contains one parentheses for each natural number. There is some infinite subset S which is the $(^n$. One can add w (a closed parentheses equal to the number of either string x open parentheses or string y open parentheses) and one is in the language and another is not in the language.  Consequently X is distinguishable from y. Therefore, by the Myhill-Nerode theorem, L is not regular. 

\paragraph{ii}
Online

\paragraph{iii}
S ? (A | ?
A ? )S | (B
B ? )A | (C
C ? )B | (D
D ? )C

\section*{Problem 3: Subsets of Regular Languages}
Let $\Sigma = \{a, b\}$ and let $L = \{ w \in \Sigma^* \ | \ w \text{ has the same number of a's and b's} \}$. 
\paragraph{i.} 
\begin{thm} $L$ is not a regular language. \end{thm}

\begin{proof}
Let $S \subseteq \Sigma^*$ and define $S = \{a^n \ | \ n \in \mathbb{N} \}$. This set is infinite because it contains one string for every natural number. Now, consider any strings $a^m, a^n \in S$ where $a^m \not = a^n$. Then $a^n b^n \in L$ and $a^m b^n \not \in L$. Consequently, $a^n \not \equiv_L a^m$. Therefore, by the Myhill-Nerode theorem, $L$ is not regular.

Note that the structure of this proof was taken from slide 41 of Small16.pdf.
\end{proof}

\paragraph{ii.}
The language $L' = \{ (ab)^n \ | \ n \in \mathbb{N} \}$ is a subset of $L$. This language is infinite because it contains one string for every natural number. However, this language is regular because it can be defined using the regular expression \texttt{(ab)*}. 

\paragraph{iii.}
\begin{thm} There is no language $L' \subseteq  \{a^nb^n \ | \ n \in \mathbb{N} \}$ that contains infinitely many strings and is a regular language. \end{thm}
\begin{proof}
In the class slides we proved that the language $L = \{a^nb^n \ | \ n \in \mathbb{N} \}$ is irregular by showing that the language is infinite and that all of the strings in the language are pairwise distinguishable. Because all of the strings in this language are pairwise distinguishable, we can choose any subset of $L$ and we know that all of the strings in that subset will be pairwise distinguishable. We choose an arbitrary infinite subset of $L$ called $S'$. Since $S'$ is both infinite and all the strings in it are pairwise distinguishable, by the Myhill-Nerode theorem we know it is an irregular language. If any arbitrary infinite subset of $L$ is irregular, then we know that there is no language $L' \subseteq  = \{a^nb^n \ | \ n \in \mathbb{N} \}$ that contains infinitely many strings and is a regular language.
\end{proof}

\section*{Problem 4: State Lower Bounds}
\paragraph{i}
let k be the number of strings in Set S. Since every string is distinguishable from one another, then each must have a different state. Therefore, since you must have the same number of states as strings, and the number of strings are the number of elements in L, then the number states is equal to the cardinality of L. 

\paragraph{ii}
Only 4 distinguishable states in the DFA. Take the set {a,aa,aaa,aaaa} these are all distinguishable in the set. For one of the strings you can add the w that adds up to 6, then other element will not be in the language. For example take a and aa. If w = aaaaa, then a will be accepted by the language but aa will not. No elements in the set have a cardinality differences of 4, so they are all distinguishable. 

\section*{Problem 5: Closure Properties Revisited}
\paragraph{i.}
\begin{thm} The nonregular languages are closed under complementation. \end{thm}
\begin{proof}
By contradiction. Assume for the sake of contradiction that there exists some nonregular language $L$ whose complement is a regular language $L'$. From the class slides, we know that the complement of any regular language is a regular language. So, because $L'$ is a regular language, its complement $L$ must also be a regular language. However we defined $L$ as nonregular. This is a contradiction, which means our assumption that there exists some nonregular language $L$ whose complement is a regular language must be false. So the nonregular languages are closed under complementation.
\end{proof}

\paragraph{ii.}
\begin{thm} The nonregular languages are not closed under union. \end{thm}
\begin{proof} We will show that there are two nonregular languages whose union is regular. Let $\Sigma = \{a, b\}$ and $L = \{a^nb^n \ | \ n \in \mathbb{N} \}$. We know that $L$ is nonregular. Let $L'$ be the complement of $L$. From the previous proof we know that $L'$ must be nonregular as well. The union of these two languages is the language that accepts any string in $\Sigma^*$: if the string is not accepted by $L$, then by definition of complementation it must be accepted by $L'$. This language can be represented by a DFA with a single accepting state that is also the start state with self-transitions on both $a$ and $b$. Because this language can be represented by a DFA, this means it is regular. Thus, the nonregular languages are not closed under union.
\end{proof}

\paragraph{iii.}
\begin{thm} The regular languages are not closed under infinite union. \end{thm}
\begin{proof}
We will prove that it is possible to construct a nonregular language from an infinite union of regular languages, proving that the regular languages are not closed under infinite union. We can define an infinite set of regular languages $L^n = \{a^n b^n\}$ for every natural number. Because each of these languages only accepts a single string, we know that each can be defined by a regular expression, making each a regular language. However, the union of these languages is $L = \{ a^n b^n \ | \ n \in \mathbb{N} \}$. We know from an earlier proof that this language is nonregular. This proves that the regular languages are not closed under infinite union.
\end{proof}

\section*{Problem 6: Designing CFGs}
\paragraph{i}
S ? ? | XaSaX
X ? ? | bX | cX | aX

\paragraph{ii}
I spent way too long on this. But it is wrong...
S ? ? | XaSbY | bSa
X ? ? | aXb | Xa
Y ? ? | aYb | Yb

S ? baX | ZabY | ?
X ? Xa | XbaY | bbX | abW | ?
Z ? Zab | Za | ?
Y ? bY | ?
W ? ab

\paragraph{iii}


\paragraph{vi}
Start symbol: S
S ? aSXXX | bXXX | SXXXX
X ? a | b

\paragraph{v}
S ? ydS | Sdy | dSy | ySd | ZySyZ | XdSdX | ?
Z ? d
X ? y
\end{document}
%%% Local Variables:
%%% mode: latex
%%% TeX-master: t
%%% End:
