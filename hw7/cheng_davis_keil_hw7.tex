%Modified from a template provided by Jennifer Pan, August 2011

\documentclass[10pt,letter]{article}
	% basic article document class
	% use percent signs to make comments to yourself -- they will not show up.
\usepackage{pdfsync}
\usepackage{amsmath}
\usepackage{amssymb}
\usepackage{amsthm}
\usepackage{ mathrsfs }
	% packages that allow mathematical formatting


\usepackage{graphicx}
	% package that allows you to include graphics
\graphicspath{ {./images/} }

\usepackage{setspace}
	% package that allows you to change spacing

\onehalfspacing
	% text become 1.5 spaced

\usepackage{fullpage}
% package that specifies normal margins

\usepackage[parfill]{parskip}

\usepackage{cancel}

\newtheorem*{thm}{Theorem}
\newtheorem{nthm}{Theorem}

\begin{document}
	% line of code telling latex that your document is beginning

\title{Problem Set 7: CS103}

\author{Katherine Cheng, Richard Davis, Marty Keil}

% \date{Friday April 10, 2015}
	% Note: when you omit this command, the current date is automatically included
 
\maketitle
	% tells latex to follow your header (e.g., title, author) commands.

\section*{Problem 1}
\section*{Problem 2}
\section*{Problem 3}
\section*{Problem 4}
\section*{Problem 5 (Richard)}
If we wanted to build a TM password checker, “entering your password” would correspond to starting up the TM on some string, and “gaining access” would mean that the TM accepts your string. Let $p \in \Sigma^*$ be your password. A TM that would work as a valid password checker would be a TM $M$ where $L(M) = {p}$; the TM accepts your string, and it doesn't accept anything else.

Given a TM, is there some way you could tell whether the TM was a valid password checker? Let $p \in \Sigma^*$ be your password and consider the following language: $$ L = \{ \langle M \rangle \ | \ M \text{ is a TM and } L(M) = \{p\}\}.$$ Prove that $L$ is undecidable. This means there is no algorithm that can mechanically check whether TM is suitable as a password checker.

\paragraph{i.} Suppose there is a function \texttt{bool isPasswordChecker(string program)} that accepts as input a program and returns whether or not that program only accepts the string $p$. Using the programs from lecture as a template, write the pseudocode for a self-referential program that uses the \texttt{isPasswordChecker} method to obtain a contradiction. 

\texttt{isPasswordChecker(isPasswordChecker)}

\paragraph{ii.} 

\section*{Problem 6 (Richard)}

\section*{Problem 7}
\section*{Problem 8}

\end{document}
%%% Local Variables:
%%% mode: latex
%%% TeX-master: t
%%% End:
