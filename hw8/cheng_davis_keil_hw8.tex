%Modified from a template provided by Jennifer Pan, August 2011

\documentclass[10pt,letter]{article}
	% basic article document class
	% use percent signs to make comments to yourself -- they will not show up.
\usepackage{pdfsync}
\usepackage{amsmath}
\usepackage{amssymb}
\usepackage{amsthm}
\usepackage{ mathrsfs }
	% packages that allow mathematical formatting


\usepackage{graphicx}
	% package that allows you to include graphics
\graphicspath{ {./images/} }

\usepackage{setspace}
	% package that allows you to change spacing

\onehalfspacing
	% text become 1.5 spaced

\usepackage{fullpage}
% package that specifies normal margins

\usepackage[parfill]{parskip}

\usepackage{listings}

\usepackage{cancel}

\newtheorem*{thm}{Theorem}
\newtheorem{nthm}{Theorem}

\begin{document}
	% line of code telling latex that your document is beginning

\title{Problem Set 7: CS103}

\author{Katherine Cheng, Richard Davis, Marty Keil}

% \date{Friday April 10, 2015}
	% Note: when you omit this command, the current date is automatically included
 
\maketitle
	% tells latex to follow your header (e.g., title, author) commands.

\section*{Problem 1: Closure Properties of RE}

\begin{enumerate}
\item[i.] 
bool inL1nL2(string w) \{\\
return inL1(w) \&\& inL2(w);\\
\}

Will return true when true, returns false or loops infinitely otherwise.
\item[ii.] Evaluates left to right (confirm?), so if the first function is an infinite loop, will not evaluate the second function (which might be true)
\item[iii.] 
bool imConvincedIsInL1uL2(string w, string c) \{\\
return imConvincedIsInL1(w, c) $||$  imConvincedIsInL2(w, c);\\
\}

Now won't get caught in an infinite loop.
\end{enumerate}

\section*{Problem 2: Password Checking}
\begin{enumerate}
\item[1.] 
meta level - by contradiction that password checking language is RE/verifiable. Then we can create this. now (i). there is some certificate c where imConvinced will return true, we accept(). If we accept, the language of the machine is (we have 2 levels of machines, (1) candidate password checkers, (2) looks at machines and says if it's a password checker) higher machine has said you're only going to accept p. Now, reach a contradiction. If we are a passwordchecker, just accepts always regardless of the input. input of the machine is everything.

contradiction: if it is a password checker then it accepts all inputs
\item[2.]
it'll infinitely loop 

we're guaranteed the pass checker returns p and only p

if we loop infinitely on every string, 
\item[3.]
input equal to p?
\end{enumerate}

\section*{Problem 3: Equivalent TMs}
m1 takes an input, it will accept/reject/loop
m2 has some other language
if imConvinced true, then language does not match what you have coded


\begin{enumerate}
\item[1.] REG
\item[2.] ALL
\item[3.] REG
\item[4.] REG
\item[5.] REG (regular languages closed under union)
\item[6.] R (M-N so not REG, but CFG so R)
\item[7.] RE (?)
\item[8.] ALL (there is a TM where L=\{$\emptyset$\} will loop forever, can't be RE)
\item[9.] RE (n is the certificate) (what happens when it loops? don't worry about that)
\item[10.] ALL
\item[11.] ALL
\item[12.] RE 
\end{enumerate}
if in RE, if I give you a machine, deep down I know this is in the machine, machine will halt. R if not in language, will halt, which is not guaranteed. Not in RE, I can give you (M,n) that's in the language and you can't conclusively tell me it's in the language. ex) at most length 5, run on all strings on at most length 5, if in language, will accept it. I'm giving you something in the language, if you halt

\section*{Problem 4: The Big Picture}

\section*{Problem 5: 4-Colorability}
\begin{enumerate}
\item[1.]
\item[2.]
\end{enumerate}
3COLOR polynomial time reducible to 4COLOR
\section*{Problem 6: Resolving P $\stackrel{?}{=}$ NP}

\begin{enumerate}
\item[1.] neither
\item[2.] neither
\item[3.] P = NP
\item[4.] P = NP
\item[5.] P $\neq$ NP
\item[6.] P $\neq$ NP
\item[7.] neither
\item[8.] neither
\item[9.] neither
\item[10.] neither
\item[11.] neither
\item[12.] neither
\item[13.] P = NP
\item[14.] P = NP
\item[15.] neither
\item[16.] neither
\end{enumerate}

\section*{Problem 7: The Big Picture}

\end{document}
%%% Local Variables:
%%% mode: latex
%%% TeX-master: t
%%% End:
