%Modified from a template provided by Jennifer Pan, August 2011

\documentclass[10pt,letter]{article}
	% basic article document class
	% use percent signs to make comments to yourself -- they will not show up.
\usepackage{pdfsync}
\usepackage{amsmath}
\usepackage{amssymb}
\usepackage{amsthm}
\usepackage{ mathrsfs }
	% packages that allow mathematical formatting


\usepackage{graphicx}
	% package that allows you to include graphics
\graphicspath{ {./images/} }

\usepackage{setspace}
	% package that allows you to change spacing

\onehalfspacing
	% text become 1.5 spaced

\usepackage{fullpage}
% package that specifies normal margins

\usepackage[parfill]{parskip}

\usepackage{listings}

\usepackage{cancel}

\newtheorem*{thm}{Theorem}
\newtheorem{nthm}{Theorem}

\begin{document}
	% line of code telling latex that your document is beginning

\title{Problem Set 7: CS103}

\author{Katherine Cheng, Richard Davis, Marty Keil}

% \date{Friday April 10, 2015}
	% Note: when you omit this command, the current date is automatically included
 
\maketitle
	% tells latex to follow your header (e.g., title, author) commands.

\section*{Problem 1: Closure Properties of RE}

\begin{enumerate}
\item[i.] 
bool inL1nL2(string w) \{\\
return inL1(w) \&\& inL2(w);\\
\}

Will return true when true, returns false or loops infinitely otherwise.
\item[ii.] Evaluates left to right (confirm?), so if the first function is an infinite loop, will not evaluate the second function (which might be true)
\item[iii.] 
bool imConvincedIsInL1uL2(string w, string c) \{\\
return imConvincedIsInL1(w, c) $||$  imConvincedIsInL2(w, c);\\
\}

Now won't get caught in an infinite loop.
\end{enumerate}

\section*{Problem 2: Password Checking}

\section*{Problem 3: Equivalent TMs}

\section*{Problem 4: The Big Picture}

\section*{Problem 5: 4-Colorability}

\section*{Problem 6: Resolving P \stackrel{?}{=} NP}

\section*{Problem 7: The Big Picture}

\end{document}
%%% Local Variables:
%%% mode: latex
%%% TeX-master: t
%%% End:
